
\documentclass[aspectratio=169]{beamer}
\usepackage[utf8]{inputenc}
\usepackage[T1]{fontenc}
%%%%%%%
% \usepackage{layout}
% \usepackage{lipsum}
%%%%%%%
\usetheme[% Complete settings. Default value in []
% titleimagecolor=red,       % [gray], darkgray, red, blue, green
% titleimagemargin=2mm,      % Distance [2mm]    Frame around title page image
% navigationsymbols=false,   % true   / [false]  Navigation symbols in the foot
% mathseriffont=false,       % true   / [false]  Serif / non-serif math fonts
% foot=true,                 % [true] / false    Footline or not
% nofootslidenum=false       % true   / [false]  Keep slide num even when foot=false
% footlogo=true,             % [true] / false    Put LU logo to the left of footer
% english=true,              % [true] / false    English / Swedish logo
% LTHlogo=false,             % true   / [false]  Use LTH logo instead of LU on title and end pages.
% blackenumeratenumber=true, % [true] / false    Black enumerate numbers, o.w. Lund bronze
% blackitemmark=false,       % true   / [false]  Black item marks, o.w. Lund bronze
% defaultfont=false,         % true   / [false]  Falls back to default beamer fonts
% sectionframe=true,
]{ulund}
%%%%%%%%%%%%%%%%%%%%% Layout commands 
%%%% Foot
% \ulundfootleft{\insertshortauthor}
% \ulundfootmid{\insertshorttitle}
% \ulundfootright{\insertframenumber}% {\insertframenumber:\inserttotalframenumber}
%%%% Titleimage
\titleimage{Pictures/ULUNDcolor} % Replaces the LU image. Voids option titleimagecolor
%%%%%%%%%%%%%%%%%%%%%%%%%%%%%%%%%%%
\title[B. Regnell, \today]{\selectfont Workshop om digitalisering \\ Näringslivsrådet LTH }
\author[\href{https://github.com/bjornregnell/ws-dig}{github.com/bjornregnell/ws-dig}]{%
  Prof. Björn Regnell\newline
  Vicerektor för digitalisering, LTH}
%%%%%%%%%%%%%%%%%%%%%
\usepackage{verbatim}
%%%%%%%%%%%%% Verbatim code box
\usepackage[skins,listings]{tcolorbox}
\tcbuselibrary{listingsutf8}

\usepackage{pgf-pie}

\newcommand{\TitleSlide}{\begin{frame}[plain]\titlepage\end{frame}}

\newcommand{\EndSlide}{\begin{frame}[plain]\endpage\end{frame}}


\newcommand{\Section}[1]{\titleimagecolor{red}\section{#1}}

\newcommand{\code}{\lstinline[basicstyle=\ttfamily]}

\newenvironment{Slide}[1]%
  {\begin{frame}[environment=Slide]{#1}}
  {\end{frame}}%

% \newenvironment{Slide}[2][]  /// AAARGH strange error???
%   {\begin{frame}[fragile,environment=Slide,#1]{#2}}
%   {\end{frame}}



\begin{document}

\TitleSlide

%%%%%%%%%%%%%%%

\begin{Slide}{Agenda: workshop om digitalisering}
\begin{itemize}
    \item Inleding, 10 min 
    \item Gruppdiskussioner, 30 min 
    \item Rapportering, 15 min 
    \item Avslutning
\end{itemize}
\end{Slide}


\begin{Slide}{Vad är digitalisering?}
  \begin{itemize}\small
      \item Digitalisering är processen att införa ny informationsteknologi (IT) i
      verksamheter. 
      \item Digitalisering av samhällen, organisationer och
      branscher avser \textbf{genomgripande verksamhetsomvandling} i samband
      med \textbf{ökad användning av modern IT} \\ och fortsatt övergång till
      \textbf{informationssamhället}. 
      
      \item Denna verksamhets- eller process-      digitalisering innefattar ändrade arbetsmetoder,
      organisationsprocesser, affärsmodeller, samhällsstrukturer och
      \textbf{kompetenskrav}.
    \item[]  ~\\\url{https://sv.wikipedia.org/wiki/Digitalisering}

  \end{itemize}
\end{Slide}



\begin{Slide}{Gruppdiskussioner}
  \begin{enumerate}
      \item  Hur påverkar digitaliseringen era produkter/tjänster/verksamheter i
      närtid / på sikt?
      
      \item Hur påverkar digitaliseringen ert behov av kompetens i närtid / på
      sikt?
  \end{enumerate}
\end{Slide}
  

\end{document}